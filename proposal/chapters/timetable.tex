\chapter{Timetable and planning}
\label{sec:timetable}

The project spans 20 weeks, which we divide as follows:

\bigskip
\begin{tabular}{l l l l}
    Task                    & Start     & End       & \# of weeks \\
    Exercise 1              & 19/2/2018 & 9/3/2018  & 3          \\
    Exercise 2              & 12/3/2018 & 23/3/2018 & 2          \\
    Exercise 3              & 26/3/2018 & 6/4/2018  & 2          \\
    Exercise 4              & 9/4/2018  & 20/4/2018 & 2          \\
    Exercise 5              & 23/4/2018 & 4/5/2018  & 2          \\
    Exercise 6              & 7/5/2018  & 11/5/2018 & 1          \\
    Exercise 7              & 14/5/2018 & 18/6/2018 & 1          \\
    Exercise 8              & 21/6/2018 & 25/6/2018 & 1          \\
    Wrapping up and writing & 28/5/2018 & 30/6/2018 & 5          \\
    Prepare thesis defense  & 2/7/2018  & 5/7/2018  & 1          \\
    Thesis defense          & 6/7/2018  & -         & -
\end{tabular}
\bigskip

For each exercise, we are going to:

\begin{enumerate}
    \item Perform program clustering and analyze the resulting data;
    \item Identify transformations that suit our acceptance criteria;
    \item Implement them in Ask-Elle;
    \item Analyze the improvements in program clustering, considering not only the current exercise but all of them;
    \item Write a brief report highlighting relevant details so we can resort to it when writing the thesis.
\end{enumerate}

Note that the amount of weeks allocated to each exercise decreases as we advance, since we expect to build upon our previous work. Additionally, extra time is allocated to exercise 1, as it involves fixing some bugs, like that \mintinline{haskell}{double xs = map (* 2) xs} is not recognized as equivalent to \mintinline{haskell}{double = map (* 2)}.
