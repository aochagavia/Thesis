\chapter{Related work}
\label{sec:related-work}

As mentioned in Chapter \ref{sec:background}, the key characteristic of Ask-Elle is its ability to give stepwise hints and to check whether two programs are semantically equivalent. Here we provide an overview of the research related to these topics.

\section{Related work in program analysis}

We are interested in proving whether two programs are semantically equivalent. This involves obtaining information from the programs to check whether the conditions for equality hold. Gathering information about a program is the subject of program analysis, which is used pervasively in automated assessment tools \cite{2016feedbackreview}.

Since program analysis can be classified as static or dynamic, the same holds for the feedback tools that rely on it. While in the past there used to be a clear distinction between static and dynamic tools \cite{2005alasurvey}, it is nowadays less clear, as many tools are combining both approaches \cite{2016feedbackreview}. For instance, Codewebs \cite{2014codewebs} and the tool presented by Huang et al. \cite{2013huang} are based on a combination of testing and static analysis. Another example is OverCode \cite{2015overcode}, which collects data from the execution of Python programs (dynamic) and combines it with data derived from the source code (static). Ask-Elle also uses a mixed approach, since it first attempts comparison by normalization (static) and resorts to property-based testing (dynamic) whenever the former is insufficient \cite{2010askelle}.

Ask-Elle's static analysis is geared towards normalization of programs. Its purpose is to identify patterns in the code that can be rewritten in a canonical form. Currently, transformations are simple enough that little analysis is needed. For most transformations, Ask-Elle makes use of term-rewriting by equational reasoning \cite{1998termrewriting}, instead of true static analysis.

% From the tools mentioned above, Codewebs and Huang rely on AST-based comparisons instead of transformations. OverCode goes a bit further, by renaming variables based on the output of dynamic analysis, but this is still far from proper normalization.

Relevant research on automatic assessment tools that rely on program normalization include Xu and Chee's work \cite{2003transformation}, who apply this method to compare Smalltalk programs. Also interesting is the work of Wang et al. \cite{2007wang}, who implement a normalization-based tool to assess C programs. A more modern approach is given by the ITAP tutoring system \cite{2017ITAP}, targeting Python programs.

From the tools mentioned, the most interesting for our purposes is ITAP, because it also supports style-based transformations tailored to beginners. This is exactly one of the features we want Ask-Elle to support. Besides this, ITAP supports reconstructing the original AST from its normalized form to produce localized feedback, including suggestions to refactor the code.

Still, none of the tutors presented above offer stepwise feedback. In fact, a recent survey of 69 tools for learning programming by Keuning et al. \cite{2016feedbackreview} identifies only 10 that give this kind of feedback. From these, only Ask-Elle, the Prolog Tutor \cite{2004hong} and ADAPT \cite{1992adapt} (also targeting Prolog) use program normalization. ELM-ART \cite{2001elmart} (targeting LISP) uses program transformations, though not for normalization purposes. It is unclear whether the approach taken by the Prolog Tutor and ADAPT can be a source of inspiration for Ask-Elle, given the big differences between Haskell and Prolog.

% \section{Related work in stepwise feedback generation}

% \subsection{Model tracing}

% Model tracing is a general purpose technique used by interactive tutoring systems to generate feedback on the process that a student is following. Student steps are compared to production rules... FSM

% ACT Programming Tutor (APT) - TPS - MT AT
% Bridge - EC TPS - MT
% DISCOVER - EC TPS - MT
% The LISP tutor - EC TPS - MT
% Ask-Elle - TPS - MT AT PT
% PATTIE - TPS - MT Other

% \subsection{Intention-based diagnosis}

% ADAPT - TPS - PT IBD
% ELM-PE / ELM-ART (II) - EC TPS - AT PT IBD
% (Hong04) - EC TPS - PT IBD

% Relies on a knowledge base. \cite{1984intention}

% \subsection{Custom diagnosis}

% ProPL - EC TPS - Other

% Dialogue and pseudocode... Really original

\section{Related work in program unification}
\label{sec:related-work-unification}

Program unification consists of comparing two programs to determine whether they are equivalent under a particular definition of equivalence. In the context of Ask-Elle, we are interested in semantic equivalence of Haskell programs, which is undecidable.

\subsection{Unification in the simply-typed lambda calculus}

The problem of unifying simply-typed lambda terms is known in the literature as \emph{higher-order unification}. While undecidable in general \cite{2013barendregt}, higher-order unification becomes decidable by imposing certain restrictions on the lambda calculus itself. A well-known example is Miller's pattern unification \cite{1991miller}, who achieves decidability by restricting beta-reduction.

\subsection{Unification in Haskell}

The Haskell language can be seen as a user-friendly version of the lambda calculus, with a series of extensions and syntactic sugar to facilitate programming. Since unification of simply-typed lambda terms is undecidable \cite{2013barendregt}, the problem of unifying Haskell terms is undecidable as well.

To visualize how Haskell is an extension of the simply-typed lambda calculus we can look at its underlying hierarchy of languages. The simply-typed lambda calculus is at the base, followed by System F \cite{2002pierce}, System FC \cite{2007systemfc}, GHC Core \cite{2007systemfc} and Haskell itself. Each language in this hierarchy is an extension of the previous one. Transitively, this means that Haskell is an extension of the simply-typed lambda calculus.

\subsection{Unification in turing-complete languages}

Unification is also undecidable for turing-complete languages that are not based in the lambda calculus. As a proof, let us first assume there is a decidable unification procedure for turing-complete languages. It is possible to write a program wich we know never halts and attempt to unify it against a second program. If unification succeeds, we know for certain that the second program never halts. Otherwise, we know for certain that it does not. In other words, a decidable unification algorithm for turing-complete languages amounts to solving the halting problem, which we know to be impossible.

\subsection{Decidable alternatives}

Given undecidability of unification, programming tutors reach to decidable alternatives. Most of them give up on true semantic equivalence and go for relaxed definitions of equivalence instead \cite{2016feedbackreview}. A clear example is Wang's grading tool for C programs \cite{2007wang}. After doing normalization, it compares a program to the model solution based on metrics like program size and control flow statements used.

The main drawback of relaxed notions of equivalence is that they do not guarantee semantic equivalence. In the case of Ask-Elle, this approach would introduce the risk of giving incorrect hints to the users. Clearly, this is not a satisfying solution for a programming tutor with its main focus on stepwise feedback.

The alternatives that restrict the language, such as Miller's \cite{1991miller}, are too inpractical to consider, as they would require rejecting valid Haskell programs. Futhermore, most research on enabling higher-order unification by restricting the language is done at the level of the lambda calculus, while Haskell provides much higher-level constructs. The interaction between those restrictions and Haskell's features is uncharted territory, out of the scope of this research.

Another alternative is to attempt normalization-based unification. This approach turns the unification problem into a normalization problem so it becomes decidable. Under this unification scheme, two programs are considered equivalent if they both normalize to the same normal form. This notion of equivalence guarantees semantic equivalence as well, though it is incomplete, meaning that for some programs it is impossible to know whether they are equivalent. This is the approach followed by Ask-Elle \cite{2010askelle}.

% Different approaches have been tried to determine whether two lambda terms are equivalent, each one with its own trade-offs. For instance, Grabmayer and Rochel \cite{2014letrec} describe an advanced mechanism based on program transformations and graph comparison. We expect to find inspiration in this kind of research to improve Ask-Elle's own normalization.

% \item Constraint-based unification. This approach relaxes the notion of equivalence so it becomes decidable. Two programs are considered equivalent if they satisfy the same set of constraints. Constraint-based unification is complete but unsound, meaning that you can compare any two programs, but that this kind of equivalence does not imply semantic equivalence. It is typically used in constraint-based programming tutors, such as [FIXME: cite?]

% \section{Related work in linting}

% More interesting to our purposes is HLint \cite{HLint}, a tool that analyzes Haskell programs and reports ways in which style can be improved. It includes a broad corpus of style-refactorings, from detecting list anti-patterns to suggesting using foldr whenever certain recursion strategies are detected. Ask-Elle can draw a lot of inspiration from HLint on the area of style-aware feedback and can even go a step further, by giving hints on programs that contain holes.

% \subsection{Other approaches to unification}

% Unification by evaluation
