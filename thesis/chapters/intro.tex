\chapter{Introduction}
\label{sec:intro}

Ask-Elle \cite{2017askelle} is a programming tutor designed to help students learn the Haskell programming language. With Ask-Elle, teachers can specify exercises along with \emph{model solutions} and students can solve them interactively. During the process, the system is able to check whether a student is on the right track and, in case they get stuck, it can provide relevant \emph{hints}. Afterwards, it can check whether the provided solution is correct.

The exercises supported by Ask-Elle ask to implement functions. For instance, an exercise to teach basic concepts around lists and recursion could be to implement the \texttt{length} function.

One of the strengths of Ask-Elle is its ability to provide feedback and hints based only on a model solution. The teacher writes one or more solutions for the exercise and Ask-Elle does the rest. This is very convenient, because it minimizes the work to set up the exercises.

\section{Ask-Elle in action}
\label{sec:intro-askelle-example-session}

To understand how Ask-Elle works, consider a hypothetical situation where a student follows Ask-Elle's hints until reaching a solution to the assignment. In our example, the task is to implement the function \haskell{double :: [Int] -> [Int]}, which multiplies each number in a list by two. We assume the exercise to have a single model solution, defined as \haskell{double = map (* 2)}.

Figure \ref{fig:interactive-session-double} shows a student's path to a solution. It starts with an empty program, which is refined step by step. \emph{Refinement} here refers to filling a hole in such a way that the resulting program is closer to a model solution. Ask-Elle hints are nothing more than refinement steps that have been generated from the model solution.

\begin{figure}
\begin{minted}{haskell}
-- Starting point (? represents a hole)
?

-- Apply hint: Introduce the function double
double = ?

-- Apply hint: Use the higher-order map function
double = map ?

-- Apply hint: Use the times operator (*)
double = map (* ?)

-- Apply hint: Introduce the integer 2
double = map (* 2)
\end{minted}
\caption{Interactive Ask-Elle session}
\label{fig:interactive-session-double}
\end{figure}

\section{Deviations from the model solution}
\label{sec:intro-fundamental-limitation}

A fundamental limitation of Ask-Elle is its inability to produce hints for programs that \emph{deviate} from the model solution. In such cases, Ask-Elle returns an error message: \emph{You have drifted from the strategy in such a way that we can not help you any more}. With such a message, the only way forward is to start over from the very beginning (an empty program, represented as \texttt{?}) or to restore a previous version of the program known to be accepted by Ask-Elle.

While the session from Figure \ref{fig:interactive-session-double} starts with the empty program, a student has the freedom to choose a different starting point. Consider, for instance, the case of a student who wants to implement a recursive version of \texttt{double}. Figure \ref{fig:limitations-recursive-double} shows a recursive implementation, which returns the same result as the model solution for all possible inputs (we call this semantic equivalence, see Section \ref{sec:bg-program-equivalence} for more details). Let us imagine that the student does not start with an empty program, but with a partial implementation of the function. What would an Ask-Elle session look like in this case? Figure \ref{fig:limitations-askelle-example-session} shows an expected yet unfortunate outcome: there are no hints available.

The main way to deal with this limitation is by defining multiple model solutions. For instance, adding a recursive model solution would fix the problem in this concrete example. Still, this is not a scalable approach, since it requires a teacher to foresee all paths a student might follow. Furthermore, there are just too many ways to drift from the model solution in terms of syntax.

\begin{figure}
\begin{minted}{haskell}
double [] = []
double (x:xs) = x * 2 : double xs
\end{minted}
\caption{Recursive implementation of \texttt{double}}
\label{fig:limitations-recursive-double}
\end{figure}

\begin{figure}
\begin{minted}{haskell}
-- Starting point
double [] = ?
double (x:xs) = ?

-- Ask-Elle's reply: You have drifted from the strategy in
-- such a way that we can not help you any more
\end{minted}
\caption{Unsuccessful Ask-Elle session}
\label{fig:limitations-askelle-example-session}
\end{figure}

\section{Research questions}
\label{sec:research-questions}

A typical use case for Ask-Elle is to aid teaching in an introductory course on functional programming. In that context, we have observed that Ask-Elle is often unable to handle student programs, regardless of whether the program is semantically equivalent to the model solution (see Chapter \ref{sec:analysis-results}). In practice, this means that Ask-Elle is in many cases failing to provide hints or to assess the correctness of a finished program.

Previous research by Gerdes et al \cite{2010askelle, 2017askelle} shows that part of the problem lies in limitations of Ask-Elle's normalization procedure (see Chapter \ref{sec:background} for details on the inner workings of Ask-Elle). The authors suggest implementing additional semantics-preserving transformations as a possible solution to the problem. This gives rise to the following questions:

\begin{enumerate}
    \item In which cases does normalization not work as expected?
    \item How can we improve Ask-Elle's normalization procedure?
\end{enumerate}

This thesis is structured as follows. Chapter \ref{sec:background} presents the necessary background knowledge about Ask-Elle's architecture. Chapter \ref{sec:related-work} refers to related work relevant to our research. Chapter \ref{sec:methodology} explains our research methodology. Chapter \ref{sec:analysis-results} identifies the main normalization issues and contributes solutions in most cases. Chapter \ref{sec:improvements} shows the measured results achieved through normalization improvements. Finally, Chapter \ref{sec:conclusion} summarizes our research and offers future work perspectives.

